%%%%%%%%%%%%%%%%%%%%%%%%%%%%%%%%%%%%%%%%%%%%%%%%%%%%%%%%%%%%%%%%
%                                                              %
% Khoi Nguyen                                                  %
% ECE 351                                                      %
% Lab 2                                                     %
% 16 September 2021                                             %
%%%%%%%%%%%%%%%%%%%%%%%%%%%%%%%%%%%%%%%%%%%%%%%%%%%%%%%%%%%%%%%%
\documentclass[11pt,a4]{article}
\usepackage[utf8]{inputenc}
\usepackage{fullpage}
\usepackage{hyperref}
\usepackage{listings}
\usepackage{xcolor}
\usepackage{graphicx}
\usepackage{float}
\definecolor{codegreen}{rgb}{0,0.6,0}
\definecolor{codegray}{rgb}{0.5,0.5,0.5}
\definecolor{codeblue}{rgb}{0,0,0.95}
\definecolor{backcolour}{rgb}{0.95,0.95,0.92}
\lstdefinestyle{mystyle}{
	backgroundcolor=\color{backcolour},
	commentstyle=\color{codegreen},
	keywordstyle=\color{codeblue},
	numberstyle=\tiny\color{codegray},
	stringstyle=\color{codegreen},
	basicstyle=\ttfamily\footnotesize,
	breakatwhitespace=false,
	breaklines=true,
	captionpos=b,
	keepspaces=true,
	numbers=left,
	numbersep=5pt,
	showspaces=false,
	showstringspaces=false,
	showtabs=false,
	tabsize=2
}
\lstset{style=mystyle}
\graphicspath{{./images/}}

\title{ECE 351 - Lab 2}
\author{Khoi Nguyen \\ https://github.com/3khoin}
\date{\today}

\begin{document}
\maketitle
\pagebreak

\tableofcontents
\pagebreak

\section{Introduction}
This lab was conducted as an introduction to user-defined Python functions (used for step and ramp function definitions). These functions were utilized to demonstrate the graphical effects of signal operations.

\section{Equations}
Step and ramp functions:
\[u(t) = {0(t \leq 0)}, 1(t \geq 0)\] 
\[r(t) = {0(t \leq 0)}, t(t \geq 0)\] 

Part 2 equation:
\[y(t) = r(t) - r(t-3) + 5u(t-3) - 2u(t-6) - 2r(t-6)\]

\section{Methodology}
The arange, cos, and zero functions were imported from the numpy library, while all of the plot functions were imported from the matplotlib.pyplot library.

We first imported the cos function from the numpy library and implemented it into a user-defined function func1 to plot a cosine wave in the domain of $0\leq t \leq 10s$. We then derived an equation for the following plot:

\begin{center}
	\includegraphics{figure1}
\end{center}

We created user-defined functions for the step and ramp functions, then another user-defined function to implement the derived equation, and plotted it from $-5\leq t \leq 10s$.

Afterwards, we performed multiple time-shifting and scaling operations, with the following inputs of t: t = \{-t, t - 4, -t - 4, t/2, 2t\}. We then hand-plotted the derivative of the user-defined function for the previously pictured plot, then implemented its derivative graphically into Python with the diff function from numpy.

\section{Results}
\begin{figure}[H]
	\centering
	\includegraphics[scale=0.4]{part1plot1.png}
	\caption{Part 1, Task 2}
\end{figure}
\begin{figure}[H]
	\centering
	\includegraphics[scale=0.4]{part2plot1.png}
	\includegraphics[scale=0.4]{part2plot2.png}
	\caption{Part 2, Task 2, 3}
\end{figure}
\begin{figure}[H]
	\centering
	\includegraphics[scale=0.4]{part3plot1.png}
	\includegraphics[scale=0.4]{part3plot2.png}
	\includegraphics[scale=0.4]{part3plot3.png}
	\includegraphics[scale=0.4]{part3plot4.png}
	\includegraphics[scale=0.4]{handplot.png}
	\includegraphics[scale=0.4]{part3plot5.png}
	\caption{Part 3, Task 2, 3, 4, 5}
\end{figure}

\section{Error Analysis}
One problem I encountered was configuring numpy.diff() to produce a somewhat relevant differentiation graph. Changing the step size and shifting the time axis somewhat helped solve this.

\section{Questions}
\begin{enumerate}
	\item The plots are not identical, because the numpy.diff function takes the difference between array members by step, so the graph will not be entirely contiguous.
	\item Changing the step size brings the numpy.diff() plot closer to the hand-drawn plot because the increase in step size means that 
	\item Decoding the documentation for numpy.diff() was somewhat confusing.
\end{enumerate}
\section{Conclusion}

\end{document}