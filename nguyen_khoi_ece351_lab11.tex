%%%%%%%%%%%%%%%%%%%%%%%%%%%%%%%%%%%%%%%%%%%%%%%%%%%%%%%%%%%%%%%%
%                                                              %
% Khoi Nguyen                                                  %
% ECE 351                                                      %
% Lab 11                                                       %
% 18 November 2021                                             %
%%%%%%%%%%%%%%%%%%%%%%%%%%%%%%%%%%%%%%%%%%%%%%%%%%%%%%%%%%%%%%%%
\documentclass[11pt,a4,titlepage]{article}
\usepackage[utf8]{inputenc}
\usepackage{fullpage}
\usepackage{hyperref}
\usepackage{listings}
\usepackage{xcolor}
\usepackage{graphicx}
\usepackage{float}
\usepackage{amsmath}
\usepackage{amssymb}
\definecolor{codegreen}{rgb}{0,0.6,0}
\definecolor{codegray}{rgb}{0.5,0.5,0.5}
\definecolor{codeblue}{rgb}{0,0,0.95}
\definecolor{backcolour}{rgb}{0.95,0.95,0.92}
\lstdefinestyle{mystyle}{
	backgroundcolor=\color{backcolour},
	commentstyle=\color{codegreen},
	keywordstyle=\color{codeblue},
	numberstyle=\tiny\color{codegray},
	stringstyle=\color{codegreen},
	basicstyle=\ttfamily\footnotesize,
	breakatwhitespace=false,
	breaklines=true,
	captionpos=b,
	keepspaces=true,
	numbers=left,
	numbersep=5pt,
	showspaces=false,
	showstringspaces=false,
	showtabs=false,
	tabsize=2
}
\lstset{style=mystyle}
\graphicspath{{./images/}}

\title{ECE 351 - Lab 11}
\author{Khoi Nguyen \\ https://github.com/3khoin}
\date{18 November 2021}

\begin{document}
\maketitle
\pagebreak

\tableofcontents
\pagebreak
	
\section{Introduction}
The goal for this lab was to utilize Python's built-in functions as well as a provided function to perform analysis on a discrete system.

\section{Equations}
\[y[k] = 2x[k] - 40x[k-1] + 10y[k-1] - 16y[k-2]\]

\section{Methodology}
Using the causal discrete function listed in the Equations section with y[k] as the output and x[k] as the input, with the system at rest, we derived the expression for H(z) = $\frac{Y(z)}{X(z)}$. From the H(z) expression, we used partial fraction expansion and the inverse Z-transform to find h[k].

\[Y(z) = 2X(z) - 40z^{-1}X(z) + 10z^{-1}Y(z) - 16z^{-2}Y(z)\]

\[Y(z)(1 - 10z^{-1} + 16z^{-2} = X(z)(2 - 40z^{-1}))\]

\[H(z) = \frac{Y(z)}{X(z)} = \frac{2 - 40z^{-1}}{1 - 10z^{-1} + 16z^{-2}} = \frac{2z^{2} - 40z}{z^{2} - 10z + 16}\]

\[\frac{H(z)}{z} = \frac{2z - 40}{(z-2)(z-8)}\]

\[A = \frac{2z - 40}{z - 8}\rvert_{z = 2} = 6, B = \frac{2z - 40}{z - 2}\rvert_{z = 8} = -4\]

\[\frac{H(z)}{z} = 6\frac{1}{z - 2} - 4\frac{1}{z - 8}; H(z) = 6\frac{z}{z - 2} - 4\frac{z}{z - 8}\]

\[h[k] = Z^{-1}\{H(z)\} = (6 * 2^{k}u[k] - 4 * 8^{k})u[k]\]

We then verified the values we obtained for A and B in our partial fraction expansion using the scipy.signal.residuez() function$^{[1]}$.

A zplane() function was provided for obtaining the pole-zero plot for $\frac{H(z)}{z}$. We afterwards plotted the magnitude and phase responses of $\frac{H(z)}{z}$ using the scipy.signal.freqz() function, with the parameter whole = True.

\section{Results}
The pole-zero plot for $\frac{H(z)}{z}$ is shown below, with poles at 2 and 8, and a zero at 20, which matches the equation for the system.
\begin{figure}[H]
	\centering
	\includegraphics[scale=0.6]{plot1}
\end{figure}

Below are the magnitude and phase responses of $\frac{H(z)}{z}$ plotted.

\begin{figure}[H]
	\centering
	\includegraphics[scale=0.6]{plot2}
\end{figure}

\section{Error Analysis}
The lab was straightforward with little room for error.

\section{Questions}
\begin{itemize}
	\item H(z) is not stable. Looking at the zero-pole plot, the poles are completely outside of the unit circle, in which all stable pole values are contained. We can further verify this with examination of the equation h[k], in which the k-exponent values are 2 and 8, which will increase to infinity over time.
	\item The clarity of the lab tasks, expectations and deliverables was good as per usual.
\end{itemize}

\section{Conclusion}
This lab focused on using Z-transforms and Python tools to perform analysis on a discrete system. In addition to verifying our hand calculations, we used Python tools to graphically confirm the behavior of our discrete system; in this case, we saw that the transfer function would be unstable.

\section{Appendix}
\begin{lstlisting}[language=Python]
 r: [ 6. -4.]
p: [2. 8.]
k: []
\end{lstlisting}

\end{document}